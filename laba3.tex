\documentclass[10pt,twocolumn]{article}
\setcounter{page}{179}
\usepackage[top=2cm,bottom=2.5cm,left=2cm,right=2cm]{geometry}
\newcommand{\RomanNumeralCaps}[1]
    {\MakeUppercase{\romannumeral #1}}
\title{\huge\textbf{ Application of an integration platform for ontological 
model-based problem solving using
an unified semantic knowledge representation}\\\vspace{1em}
\large Valerian Ivashenko\\\small{Department of Intellectual Information Technologies}\\\textit{Belarusion State University of Informatics and Radioelectronics}\\Minsk, Republic of Belarus\\ivashenko@bsuir.by\vspace{-3em}}
\date{}
\begin{document}
\maketitle
\textbf{\small {\textit {Abstract}—This article describes a solution in the form
of an intelligent integration platform based on the model
of an unified semantic knowledge representation for the
development of applied knowledge-driven systems. The
model of an unified semantic knowledge representation
using semantic networks, models and methods of measure
and probability theories, methods of descrete optimization
and applied mathematics, computer simulation and multiagent approaches were used. The purpose is to develop computer tools with cognitive architecture relying on elements
of artificial consciousness and being able to communicate
and to be flexible and adaptive in complex educational
applications. Virtual machines, subsystems of integration
platform and tutoring applied multi-agent software system were developed and implemented as part of humanmachine interaction system.}}


\textbf{\small{\textit{Keywords}—artificial intelligence systems integration, integration platform, multi-agent system, knowledge-driven
system, knowledge processing model, unified semantic
knowledge processing model}\vspace{-0.7em}}

\begin{center}
\section*{\large\textmd{{\RomanNumeralCaps{1.}  Introduction}}}
\end{center}

\small The basis for the success of a learning intellectual
system is integration openness [42], [45]. When it is
possible to integrate not only new knowledge related to
different types but also various mechanisms for solving
problems. Integration is one of the important understanding mechanisms for knowledge systems allowing the
acquisition and improvement of problem-solving skills
which is important for knowledge-driven systems [42],
[45]. This article presents a solution in the form of
an integration platform based on the unified semantic
knowledge representation model. This presentation attempts to answer the following questions.
\begin{enumerate} 
  \item What goals can be achieved using such platforms?
   \vspace{-0.2em}
  \item  What developments are already in this direction?
   \vspace{-0.2em}
  \item   What are the architecture, mechanisms and rules for using the platform?
   \vspace{-0.2em}
  \item  What are the positive and negative peculiarities of the platform?
   \vspace{-0.2em}
  \item  What results have been achieved in the process of its application?
   \vspace{-0.2em}
  \item  What are the perspectives for the development of the platform?
\end{enumerate}

The general goals planned to be achieved are:

\begin{itemize}
  \item creation of dynamically updating knowledge-based
system able to accept new knowledge via machine
learning and high-level or natural language communication;
 \vspace{-0.2em}
  \item creation of scalable knowledge-driven systems
maintaining big knowledge and large scale integrated ontology;
  \item creation of artificial consciousness systems which
are self-descriptive and introspective;
 \vspace{-0.2em}
  \item creation of multi-agent distributed applied intellectual systems.
\end{itemize}
\begin{center}
\section*{\large\textmd{{\RomanNumeralCaps{2.}}  Overview of models and approaches}}
\vspace{-0.5em}
\end{center}


The necessity of artificial intelligence integration historically caused by the existence of separate artificial
intelligence solutions for specific problems such as reasoning knowledge, speech synthesis and recognition,
computer vision problems. General approaches to the
integration of information systems: integration through
translation with control passing, integration through interpretation or through communication without control
passing. In the case of control passing each process or
agent of system should store own state. Therefore, the
corresponding storage is available for one control flow
and both are shared by all processes or agents on the
system.


From the other side, consciousness as a more advanced
kind of intelligence is determined by social experience of
natural language communication. Thus, by natural way,
such communicative models as actor models [1], [5], [13]
or multi-agent systems are the basis [6], [45] not only
for concurrent computer system but artificial intelligence
systems. Therefore, concurrent models are integration
models. These models are divided into two classes. The
first class is models without shared common memory or
storage. The second class is models with shared common\vspace{5mm}
\end{document}

